\documentclass{article}

\usepackage{ucs}

\usepackage[utf8x]{inputenc}
\usepackage[greek, english]{babel}
\usepackage{alphabeta}
\usepackage{lmodern}

\usepackage{listings}

\usepackage{graphicx}
\graphicspath{./images/}

\usepackage{forest}

\title{Πρώτο μέρος εργαστηριακής εργασίας στο μάθημα Τεχνήτη Νοημοσύνη}
\date{2020-11-22}
\author{Ιάκωβος Μαστρογιαννόπουλος - cse242017102}

\definecolor{codegreen}{rgb}{0,0.6,0}
\definecolor{codegray}{rgb}{0.5,0.5,0.5}
\definecolor{codepurple}{rgb}{0.58,0,0.82}
\definecolor{backcolour}{rgb}{0.95,0.95,0.95}

\lstdefinestyle{mystyle} {
    backgroundcolor=\color{backcolour},
    commentstyle=\color{codegreen},
    keywordstyle=\color{magenta},
    numberstyle=\tiny\color{codegray},
    stringstyle=\color{codepurple},
    basicstyle=\ttfamily\footnotesize,
    breakatwhitespace=false,
    breaklines=true,
    captionpos=b,
    keepspaces=true,
    numbers=left,
    numbersep=5pt,
    showspaces=false,
    showstringspaces=false,
    showtabs=false,
    tabsize=2
}

\lstset{style=mystyle}

\begin{document}
    \pagenumbering{gobble}
    \maketitle
    
    \newpage
    \tableofcontents
    \newpage
    \lstlistoflistings

    \newpage
    \pagenumbering{arabic}
    \section{Πρόλογος}

    \paragraph{}
    Στη συγκεκρίμενη εργασία του μαθήματος <<Τεχνητής Νοημοσύνης>> είχαμε να μελετήσουμε το προβλήμα του parking. Βέβαια,
    για να μπορέσουμε να φτάσουμε στο σημείο της υλοποίησης του κωδικά, πρώτα πρέπει να εξηγηθούν μερικά πραγμάτα με το ποιο είναι
    το πρόβλημα, πώς μπορούμε να το υλοποιήσουμε και ποια θα είναι τα πιθανά αποτελέσματα που θα μπορούσαμε να πάρουμε πίσω.  
    Όπως έχει αναφερθεί και σε email, την εργασία την κάνει ένα ατόμο μόνο του. Η γλώσσα προγραμματισμού που υλοποιήθηκαν οι αλγόριθμοι
    είναι η Python, στον IDE Pycharm και συγκεκριμένα ήταν η έκδοση 3.8.6.

    \newpage
    \section{Κατανόηση προβλήματος - Προβλήμα του Parking}
    \paragraph{}
    Το πρόβλημα που έχουμε να λύσουμε είναι το πρόβλημα του Parking. Θεωρετηκά, υπάρχει ένας αυτόματος οδηγός ο οποίος προσπαθεί να βρει
    ελευθέρο χώρο για να γεμίσει τα αμάξια που θέλουν να μπουν μέσα στο Parking. Υπάρχουν Ν spaces από τα οποία θα μπορούσαν μερικά από αυτά
    να ήταν τελείως άδεια, ενώ κάποια αλλά να είχαν μια πλατφόρμα.

    \paragraph{}
    Σκοπός είναι να βρίσκει τις πλατφόρμες και από εκεί να καταλαβαίνει ποιες από αυτές έχουν ελεύθερο χρόνο και να τις γεμίζει. Τα βήματα που
    πρέπει να ακολουθήσουμε είναι τα εξής:

    \begin{itemize}
        \item Να βρίσκει σε ποιο node με πλατφόρμα υπάρχει ελεύθερη θέση
        \item Να τρέχει έναν αλγόριθμο αναζήτησης και να βρίσκει το πιο γρήγορο path
        \item Να ανταλλάζει θέση τα nodes μεταξύ τους, έτσι ώστε το node με την άδεια πλατφόρμα να πηγαίνει στην θέση 1
        \item Τέλος να έχει κάποιον στόχο που όταν τον εκπληρώσει να λήγει το πρόγραμμα
    \end{itemize}

    \newpage
    \section{Μοντελοποίηση του προβλήματος}
    \subsection{Χώρος καταστάσεων}
    \paragraph{}
    Χώρο καταστάσεων ενός προβλήματος ονομάζουμε το σύνολο των πιθανών καταστάσεων, στις οποίες μπορούν να βρεθούν οι καταστάσεις του προβλήματος.
    Στο πρόβλημα του parking τα αντικείμενα είναι τα αυτοκίνητα, τα spaces και οι πλατφόρμες. Μια κατάσταση είναι εάν το space έχει πλατφόρμα ή εάν η
    πλατφόρμα έχει ελεύθερο χώρο για να χωρέσει αυτοκίνητα.

    \subsection{Αρχική κατάσταση}
    \paragraph{}
    Η αρχική κατάσταση του Parking απαιτεί να υπάρχουν 4 spaces, όπου μόνο 3 από αυτά έχουν ελεύθερες θέσεις, και 3 αυτοκίνητα που περιμένουν απέξω για 
    να μπουν μέσα. Στην προέκταση, μας ζητήθηκε να αυξήσουμε τον αριθμό των spaces.

    \subsection{Τελική κατάσταση}
    \paragraph{}
    Δεν υπάρχει κάποια συγκεκριμένη τελική κατάσταση για το συγκεκριμένο πρόβλημα. Ομώς, εμείς μπορούμε να θεωρήσουμε ως τελική κατάσταση όταν όλες οι πλατφόρμες
    έχουν γεμίσει με αυτοκίνητα.

    \subsection{Τελεστές προβλήματος}
    \paragraph{}
    Στο πρόβλημα μας έχουμε δύο τελεστές:

    \begin{itemize}
        \item Τελεστή IN: ο Τελεστής όπου βάζει τα αυτοκίνητα μέσα στις πλατφόρμες.
        \item Tελεστή Neighbour: ο Τελεστής όπου διαβάζει τους γειτονές του κάθε node.
    \end{itemize}

    Στην κωδικοποίηση, βέβαια, η υλοποιήση των τελεστών ήταν λίγο διαφορετική, αλλά η ίδεα παραμένει ίδια.

\end{document}